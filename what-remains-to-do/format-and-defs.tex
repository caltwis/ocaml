% -*- mode: latex; fill-column: 79; mode: auto-fill; mode: flyspell; buffer-file-coding-system: utf-8 -*-

% Taken from my PhD thesis sources.  Of course a lot of this is redundant here.  --L.S.

\usepackage{etex}
%\usepackage{savesym} % this provides \savesymbol{} and \restoresymbol{}{}
%\usepackage{bbding}
%\usepackage{mathfrak}
%\usepackage[open-square,define-standard-theorems]{QED}
%\usepackage{amssymb}

%\newcommand\hmmax{0} % default 3
% \newcommand\bmmax{0} % default 4

\usepackage{xfrac} % This provides \sfrac, for making slanted functions

%\usepackage{skull}
%\usepackage{ulsy}

%% %\usepackage{pbox}
%% \usepackage{float}
%% \floatstyle{boxed} 
%% \restylefloat{figure}

\usepackage{xspace}
\usepackage{amsmath}
\usepackage{amssymb}             % AMS Math
\usepackage{mathrsfs} % Used for \mathscr
\usepackage{mathabx}
\usepackage{slashed}

\usepackage{caption} % Useful for figures spanning multiple pages

%% % MnSymbols screws up some symbols I like, by redefining the entire symbol
%% % font:  I found the main idea of this workaround in a piece of code
%% % suggested by Scott Pakin on comp.text.tex:
%% %\usepackage{MnSymbol} % ... but I want some symbols defined here
%% % define a new font corresponding to what MnSymbol has overridden:
%% \DeclareSymbolFont{ltxsymbols}{OMS}{cmsy}{m}{n}
%% % Make the symbols I like come from our new font:
%% %\DeclareMathSymbol{\leq}{\mathord}{ltxsymbols}{"14}
%% %\DeclareMathSymbol{\geq}{\mathord}{ltxsymbols}{"15}
%% \DeclareMathSymbol{\bot}{\mathord}{ltxsymbols}{'077}
%% \DeclareMathSymbol{\top}{\mathord}{ltxsymbols}{'076}

\usepackage{stmaryrd}
%\usepackage{ragged2e} % This contains the justify environment
%\usepackage{amssymb}
\usepackage{url}
%\usepackage{graphicx}
%\usepackage{amssymb}
%\usepackage{amsmath}
%\usepackage{multirow}
\usepackage{multicol}
\usepackage{ucs}
\usepackage[utf8x]{inputenc}
\usepackage{listings}
\usepackage{fancyvrb} % For the Verbatim (big V) environment.
                      % Also makes
                      % \verbatim work in footnotes, if I run \VerbatimFootnotes *after* the preamble:

%% % Reset the footnote counter at each page:
%% \usepackage{perpage}
%% \MakePerPage{footnote}

%\usepackage{boxedminipage}
\usepackage{color}
%\usepackage{proof}
\usepackage{bussproofs}
\usepackage{semantic} % For T diagrams
%\usepackage{xspace} % This shouldn't be needed for English, but I may be wrong --positron
\usepackage{wasysym} % for \frownie and \lightning
\usepackage{ifsym}
%\usepackage[english]{babel}
% \usepackage[french]{babel}
%\usepackage[latin1]{inputenc}
%% %% \usepackage[T2A,T1]{fontenc} % T2A is for Cyrillic; I don't need it in my
%% %%                               % thesis, but I want to keep this information
%% %%                               % around (commented out)
%% %% \newcommand\cyrtext[1]{{\fontencoding{T2A}\selectfont #1}}
\usepackage[T1]{fontenc}
%\usepackage{keystroke} % \keystroke{foo} looks nice
%\usepackage{epstopdf}
\usepackage{enumerate}
%\usepackage{breqn} % Automatic line breaks in display math mode;
                   % but it's apparently kludgy and dangerous, and
                   % might have bad interactions
%\usepackage{algorithm}

%% % PGF craziness:
\usepackage{pgf}
\usepackage{tikz}
%\usepackage{pgffor}
\usetikzlibrary{automata,calc,through,backgrounds,arrows,decorations.pathmorphing,decorations.shapes,fit,positioning,matrix}
\tikzset{terminal/.style=...}

%\usepackage{natbib}

% My trivial and hackish higher-level conditional facility:
\newcommand{\WHENCOMPLETE}[1]{\ifx\COMPLETE\THISNAMEISDEFINITELYNOTBOUND{}\else{#1}\fi}
\newcommand{\INCLUDEWHENCOMPLETE}[1]{\WHENCOMPLETE{\include{#1}}}
\newcommand{\UNLESSCOMPLETE}[1]{\ifx\COMPLETE\THISNAMEISDEFINITELYNOTBOUND{#1}\else{}\fi}
\newcommand{\INCLUDEUNLESSCOMPLETE}[1]{\UNLESSCOMPLETE{\include{#1}}}

% My funny non-standard macros:
%% \newcommand{\QUOTAION}[2]{
%% \begin{quotation}
%% {\sf #1}

%% --- {#2}
%% \end{quotation}}
\definecolor{pagecounter}{rgb}{0.5,0.5,0.5}
\definecolor{darkred}{rgb}{0.6,0,0}
\definecolor{darkgreen}{rgb}{0,0.6,0}
\definecolor{darkblue}{rgb}{0,0,0.6}
%% \definecolor{darkred}{rgb}{0.5,0,0}
%% \definecolor{darkgreen}{rgb}{0,0.5,0}
%% \definecolor{darkblue}{rgb}{0,0,0.5}
\definecolor{darkyellow}{rgb}{0.4,0.4,0}
\definecolor{purple}{rgb}{0.4,0,1}
\definecolor{brown}{rgb}{0.6,0.1,0}
\definecolor{white}{rgb}{1,1,1}
\definecolor{black}{rgb}{0,0,0}
\def\TEST{\textcolor{darkgreen}}
\newcommand{\NOTE}[1]{\small {\textcolor{red}{[}}{\textcolor{blue}{#1}}{\textcolor{red}{]}}}
\newcommand{\urlsmall}[1]{{\scriptsize\url{#1}}}
\newcommand{\REMOVE}[1]{{\textcolor{brown}{[{\bf Remove}:~{\em #1}]}}}
\newcommand{\SUGGESTIONS}[1]{{\textcolor{brown}{[{\bf I accept suggestions}:~{#1}]}}}
\newcommand{\TOOMUCH}[1]{{\textcolor{brown}{[{\bf Is this too much?}~{#1}]}}}
\newcommand{\REREAD}[1]{{\textcolor{brown}{[{\bf Re-read}:~{#1}]}}}
\newcommand{\TODO}[1]{{\textcolor{red}{[{\bf To do}:~{#1}]}}}
\newcommand{\TODOF}[1]{\footnote{\TODO{#1}}}
\newcommand{\NEW}[1]{\textcolor{darkgreen}{\textbf{[New:} {#1}\textbf{]}}}
\newcommand{\REPHRASED}[1]{\textcolor{darkgreen}{\textbf{[Rephrased:} {#1}\textbf{]}}}
\newcommand{\RATIONALEF}[1]{\footnote{\RATIONALE{#1}}}
\newcommand{\TODOQ}[1]{{\textcolor{red}{#1}}}
\newcommand{\PREMISEWHICHCOULDBEPROVEN}[1]{}%{{\textcolor{purple}{#1}}}
\newcommand{\DONE}[1]{{\textcolor{darkgreen}{[{\bf Done}:~{#1}]}}}
\newcommand{\DONEQ}[1]{\textcolor{darkgreen}{#1}}
\newcommand{\Q}[1]{\textcolor{red}{[\textit{#1}]}}
\newcommand{\STRONG}[1]{{\textcolor{red}{[{\bf Too strong}:~{#1}]}}}
\newcommand{\STRONGQ}[1]{{\textcolor{red}{#1}}}
\newcommand{\VIOLENT}[1]{{\textcolor{red}{[{\bf Too violent}:~{#1}]}}}
\newcommand{\MAYBEVIOLENT}[1]{{\textcolor{red}{{\bf [Violent?]}~{#1}}}}
\newcommand{\VIOLENTQ}[1]{\STRONGQ{#1}}
\newcommand{\USELESS}[1]{{\textcolor{brown}{[{\bf Useless?}:~{#1}]}}}
\newcommand{\REFORMULATE}[1]{{\textcolor{brown}{[{\bf Reformulate}:~{#1}]}}}
\newcommand{\MOVE}[1]{{\textcolor{blue}{[{\bf Move}:~{#1}]}}}
\newcommand{\MOVEQ}[1]{{\textcolor{blue}{#1}}}
\newcommand{\MAYBEMOVE}[1]{{\textcolor{blue}{[{\bf Move?}~{#1}]}}}
\newcommand{\MAYBE}[1]{{\textcolor{darkgreen}{{\bf ?}{#1}{\bf ?}}}}
\newcommand{\MAYBEQ}[1]{{\textcolor{darkgreen}{#1}}}
\newcommand{\IMPORTANT}[1]{{\textcolor{blue}{[{\bf Important}:~{\em #1}]}}}
\newcommand{\REMINDER}[1]{{\textcolor{purple}{[{\bf Reminder}:~{\em #1}]}}}
\newcommand{\RATIONALE}[1]{{\textcolor{purple}{[{\bf Rationale}:~{\em #1}]}}}
\newcommand{\IMP}[1]{\IMPORTANT{#1}}
\newcommand{\CHECKINTHEEND}[1]{{\textcolor{brown}{[{\bf Check at the end}:~{\em #1}]}}}
\newcommand{\CHECK}[1]{{\textcolor{brown}{[{\bf Check}:~{\em #1}]}}}
\newcommand{\MYOR}[2]{\textcolor{red}{{\bf [}{{\textcolor{darkgreen}{#1}}{\bf ~OR~}{{\textcolor{darkgreen}{#2}}{\bf ]}}}}}
\newcommand{\ORTWO}[2]{\textcolor{red}{{\bf [}{{\textcolor{darkgreen}{#1}}{\bf ~OR~}{{\textcolor{darkgreen}{#2}}{\bf ]}}}}}
\newcommand{\ORTHREE}[3]{\textcolor{red}{{\bf [}{{\textcolor{darkgreen}{#1}}{\bf ~OR~}{{\textcolor{darkgreen}{#2}}{\bf ~OR~}{{\textcolor{darkgreen}{#3}}{\bf ]}}}}}}
\newcommand{\ORFOUR}[4]{\textcolor{red}{{\bf [}{{\textcolor{darkgreen}{#1}}{\bf ~OR~}{{\textcolor{darkgreen}{#2}}{\bf ~OR~}{{\textcolor{darkgreen}{#3}}{\bf ~OR~}{{\textcolor{darkgreen}{#4}}{\bf ]}}}}}}}
\newcommand{\SYNONYM}[1]{\textcolor{blue}{{\bf [Find a synonym}:~{\em #1}{\bf ]}}}
\newcommand{\UNSURE}[1]{\textcolor{red}{{\bf [I'm not sure of this}:~{\em #1}{\bf ]}}}
\newcommand{\WORD}[1]{\textcolor{purple}{{\bf [Is there a better term?}~{\em #1}{\bf ]}}}
\newcommand{\TERM}[1]{\WORD{#1}}
\newcommand{\LANGUAGE}[1]{\WORD{#1}}
\newcommand{\GRAMMAR}[1]{\textcolor{red}{{\bf [Is this correct in English?}~{\em #1}{\bf ]}}}
\newcommand{\GRAMMARQ}[1]{\textcolor{red}{#1}}
\newcommand{\ENGLISH}[1]{\GRAMMAR{#1}}
\newcommand{\ITALIANISM}[1]{\textcolor{red}{{\bf [Is this an italianism?}~{\em #1}{\bf ]}}}
\newcommand{\EXCESSIVE}[1]{\textcolor{red}{{\bf [This may be excessive}:~{\em #1}{\bf ]}}}
\newcommand{\HACKER}[1]{\textcolor{red}{{\bf [This is informal jargon}:~{\em #1}{\bf ]}}}
\newcommand{\JARGON}[1]{\textcolor{red}{{\bf [This is informal jargon}:~{\em #1}{\bf ]}}}
\newcommand{\INFORMAL}[1]{\textcolor{red}{{\bf [This is informal}:~{\em #1}{\bf ]}}}
\newcommand{\UGLY}[1]{\textcolor{red}{{\bf [ugly}:~{\em #1}{\bf ]}}}
\newcommand{\DISLIKE}[1]{\textcolor{red}{{\bf [I don't like this}:~{\em #1}{\bf ]}}}
\newcommand{\DONTLIKE}[1]{\DISLIKE{#1}}
\newcommand{\GENIAL}[1]{{\textcolor{darkgreen}{[{\bf A genial idea}:~{#1}]}}}
\newcommand{\GOOD}[1]{{\textcolor{darkgreen}{[{\bf Good}:~{#1}]}}}
\newcommand{\IGNORE}[1]{}
\newcommand{\OBSOLETE}[1]{{\textcolor{brown}{[{\bf Obsolete}:~{#1}]}}}
\newcommand{\OBSOLETEQ}[1]{{\textcolor{brown}{#1}}}
\newcommand{\SW}[1]{{\textcolor{darkgreen}{[{\bf Somewhere}:~{#1}]}}}
\newcommand{\SOMEWHERE}[1]{\SW{#1}}
\newcommand{\MAYBESOMEWHERE}[1]{\MAYBE{\SOMEWHERE{#1}}}
\newcommand{\SOMEWHEREMAYBE}[1]{\MAYBESOMEWHERE{#1}}
\newcommand{\FILL}[0]{{\textcolor{red}{\bf [Fill this...]}}}
\newcommand{\SOMETHING}[0]{{\textcolor{red}{\bf [Something]}}}
\newcommand{\PROBABLYNOT}[1]{{\textcolor{red}{[{\bf Probably not}:~{#1}]}}}
\newcommand{\NO}[1]{\textcolor{brown}{{\bf [I don't like this}:~{\em #1}{\bf ]}}}
\newcommand{\NOTREALLY}[1]{\textcolor{brown}{{\bf [Not really}:~{\em #1}{\bf ]}}}
\newcommand{\WHYNOT}[1]{\textcolor{red}{{\bf [Why not?}~{\em #1}{\bf ]}}}
\newcommand{\WHYNOTQ}[1]{\textcolor{red}{#1}}
\newcommand{\YES}[1]{\textcolor{red}{{\bf [Yes}:~{\em #1}{\bf ]}}}
\newcommand{\YESQ}[1]{\textcolor{red}{#1}}
\newcommand{\WRONG}[1]{\textcolor{brown}{{\bf [Wrong}:~{\em #1}{\bf ]}}}
\newcommand{\WRONGQ}[1]{\RED{#1}}
\newcommand{\LONG}[1]{\textcolor{red}{{\bf [Only for the long version}:~{\em #1}{\bf ]}}}
\newcommand{\SHORT}[1]{\textcolor{red}{{\bf [Only for the short version}:~{\em #1}{\bf ]}}}
\newcommand{\LONGSHORT}[2]{\LONG{#1}{\SHORT{#2}}}
\newcommand{\SHORTLONG}[2]{\SHORT{#1}{\LONG{#2}}}
\newcommand{\TEMP}[1]{{\textcolor{darkgreen}{[{#1}]}}}
\newcommand{\META}[1]{\textcolor{darkgreen}{[{\em {#1}}]}}
\newcommand{\CH}[1]{{\textcolor{darkgreen}{[Christophe: {\em {#1}}]}}}
\newcommand{\JV}[1]{{\textcolor{darkgreen}{[Jean-Vincent: {\em {#1}}]}}}
\newcommand{\ADVISORS}[1]{{\textcolor{darkgreen}{[Christophe et Jean-Vincent: {#1}]}}}
\newcommand{\ADV}[1]{\ADVISORS{#1}}
\newcommand{\INVISIBLE}[1]{{\textcolor{white}{{#1}}}}
\newcommand{\SKIPALINE}[0]{{\\\INVISIBLE{.}\\}}
%\newcommand{\NOTHING}[0]{}
\newcommand{\NOTHING}[0]{\INVISIBLE{\em hack}}
\def\TEMPBLOCK{\textcolor{darkgreen}}

\newcommand{\FATBRACKETS}[1]
  {\mbox{$[\![ #1 ]\!]$}}
\newcommand{\BANANABRACKETS}[1]
  {\mbox{$\llparenthesis {#1} \rrparenthesis$}}
\newcommand{\LFATBRACKETS}[2]{\mbox{$\mathcal{#1}[\![ #2 ]\!]$}}
\newcommand{\PFATBRACKETS}[2]{\mbox{$#1[\![ #2 ]\!]$}}
\newcommand{\TRANSFORM}[2]{\mbox{${#1}[\![ #2 ]\!]$}}
\newcommand{\SSYNTAX}[1]{\mbox{$S[\![ #1 ]\!]$}}

\newcommand{\LIFT}[1]
  {\lfloor{#1}\rfloor}
%  {\lvert{#1}\rvert}
%  {\lceil\lfloor{#1}\rfloor\rceil}

\newcommand{\MACROEXPANDSTO}[0]{\ \ \equiv\ \ }
\newcommand{\EQD}[0]{\triangleq}

%% Reified sementic structures:
\newcommand{\PRIMITIVE}[1]{\mathcal{P}({#1})} % primitive name
\newcommand{\FUNCTION}[3]{\mathcal{T}({#1}, {#2}, {#3})} % environment, formal, code
%\newcommand{\CONTINUATION}[3]{\mathcal{K}({#1}, {#2}, {#3})} % environment, formal, code
\newcommand{\CONTINUATION}[1]{\mathcal{K}({#1})}
\newcommand{\FUTURE}[1]{\mathcal{T}({#1})} % token
\newcommand{\TASK}[1]{\mathcal{T}({#1})} % identifier

% Tasks:
%\def\TASKFONT{\mathsf}
%\def\TASKFONT{\mathfrak}
\newcommand{\BUSYTASK}[2]{{\TASKFONT{B}}({#1},{#2})} % environment, expression
\newcommand{\READYTASK}[1]{{\TASKFONT{R}}({#1})}
\newcommand{\DEADTASK}[0]{\TASKFONT{D}}
%\newcommand{\K}[3]{\mathcal{K}({#1}, {#2}, {#3})} % environment, formal, code

%% Stuff for writing rules:
\newcommand{\tsq}[0]{\vdash}
\newcommand{\ts}[0]{\ \vdash\ }
\newcommand{\tr}[0]{\ \rhd\ }
\newcommand{\arNS}[1]{\to_{#1}}
\newcommand{\ar}[1]{\ \arNS{#1}\ }
\newcommand{\NOAXIOM}[0]{\AxiomC{\NOTHING}}
\newcommand{\NOAXIOMQ}[0]{\AxiomC{}}
\newcommand{\tto}[0]{\ \twoheadrightarrow\ }
%%
\newcommand{\TWORULES}[2]
           {\begin{minipage}[b]{\linewidth}
               \begin{minipage}[b]{0.5\linewidth}
                 \centering {#1}
               \end{minipage}
               \begin{minipage}[b]{0.5\linewidth}
                 \centering {#2}
               \end{minipage}
             \end{minipage}}
\newcommand{\THREERULES}[3]
           {\begin{minipage}[b]{\linewidth}
               \begin{minipage}[b]{0.32\linewidth}
                 \centering {#1}
               \end{minipage}
               \begin{minipage}[b]{0.32\linewidth}
                 \centering {#2}
               \end{minipage}
               \begin{minipage}[b]{0.32\linewidth}
                 \centering {#3}
               \end{minipage}
             \end{minipage}}


\usepackage{aecompl}

% Links in pdf
\usepackage{color}
\definecolor{linkcol}{rgb}{0,0,0.4} 
\definecolor{citecol}{rgb}{0.5,0,0} 

% definitions.
% -------------------

\setcounter{secnumdepth}{3}
\setcounter{tocdepth}{2}

% Some useful commands and shortcut for maths:  partial derivative and stuff

\newcommand{\pd}[2]{\frac{\partial #1}{\partial #2}}
\def\abs{\operatorname{abs}}
\def\argmax{\operatornamewithlimits{arg\,max}}
\def\argmin{\operatornamewithlimits{arg\,min}}
\def\diag{\operatorname{Diag}}
\newcommand{\eqRef}[1]{(\ref{#1})}

\usepackage{rotating}                    % Sideways of figures & tables
%\usepackage{bibunits}
%\usepackage[sectionbib]{chapterbib}          % Cross-reference package (Natural BiB)
%\usepackage{natbib}                  % Put References at the end of each chapter
                                         % Do not put 'sectionbib' option here.
                                         % Sectionbib option in 'natbib' will do.

% \usepackage{txfonts}                     % Public Times New Roman text & math font
  
%\usepackage{algorithm}
%\usepackage[noend]{algorithmic}

%%% Clear Header %%%%%%%%%%%%%%%%%%%%%%%%%%%%%%%%%%%%%%%%%%%%%%%%%%%%%%%%%%%%%%%%%%
% Clear Header Style on the Last Empty Odd pages
\makeatletter

\def\cleardoublepage{\clearpage\if@twoside \ifodd\c@page\else%
  \hbox{}%
  \thispagestyle{empty}%              % Empty header styles
  \newpage%
  \if@twocolumn\hbox{}\newpage\fi\fi\fi}

\makeatother
 
%%%%%%%%%%%%%%%%%%%%%%%%%%%%%%%%%%%%%%%%%%%%%%%%%%%%%%%%%%%%%%%%%%%%%%%%%%%%%%% 
% Prints your review date and 'Draft Version' (From Josullvn, CS, CMU)
\newcommand{\reviewtimetoday}[2]{\special{!userdict begin
    /bop-hook{gsave 20 710 translate 45 rotate 0.8 setgray
      /Times-Roman findfont 12 scalefont setfont 0 0   moveto (#1) show
      0 -12 moveto (#2) show grestore}def end}}
% You can turn on or off this option.
% \reviewtimetoday{\today}{Draft Version}
%%%%%%%%%%%%%%%%%%%%%%%%%%%%%%%%%%%%%%%%%%%%%%%%%%%%%%%%%%%%%%%%%%%%%%%%%%%%%%% 

\newenvironment{maxime}[1]
{
\vspace*{0cm}
\hfill
\begin{minipage}{0.5\textwidth}%
%\rule[0.5ex]{\textwidth}{0.1mm}\\%
\hrulefill $\:$ {\bf #1}\\
%\vspace*{-0.25cm}
\it 
}%
{%

\hrulefill
\vspace*{0.5cm}%
\end{minipage}
}

\usepackage{multirow}
\usepackage{pdflscape}
\usepackage{amsfonts} % This interferes with mathbbol and they are not
                      % commutative, but I don't understand exactly what
                      % the interaction is; anyway I don't use amsfonts


\newenvironment{bulletList}%
{ \begin{list}%
	{$\bullet$}%
	{\setlength{\labelwidth}{25pt}%
	 \setlength{\leftmargin}{30pt}%
	 \setlength{\itemsep}{\parsep}}}%
{ \end{list} }

% centered page environment
\newenvironment{vcenterpage}
{\newpage\vspace*{\fill}\fancyhf{}\renewcommand{\headrulewidth}{0pt}}
{\vspace*{\fill}\par\pagebreak}

% My funny macros
\newcommand{\NEWLINE}[0]
  {\INVISIBLE{\tiny\tiny\tiny.}\\}
%% \lstset{language=Lisp,
%%         basicstyle=\tiny\ttfamily,
%%         numbers=left,
%%         stringstyle=\ttfamily,
%%         keywordstyle=\ttfamily,
%%         commentstyle=\ttfamily,
%%         identifierstyle=\ttfamily
%% }
%        numbers=left,
\lstset{language=Lisp,
        basicstyle=\small\ttfamily,
        numbers=none,
        stringstyle=\ttfamily,
        keywordstyle=\ttfamily,
        commentstyle=\ttfamily,
        identifierstyle=\ttfamily
}
%\def\NANOLISP{\textcolor{red}{nanolisp}}
\def\NANOLISP{nanolisp\xspace}
% My hackish support for grammars:
\newcommand{\SPACER}[0]{\\\INVISIBLE{\small{\tiny{hack!}}}}
%\newcommand{\SPACERF}[0]{\SPACER$\ \ $}
\newcommand{\SPACERP}[0]{\SPACER$|\ $}
\newcommand{\SPACERF}[0]{\SPACER\INVISIBLE{$|\ $}}
%% Example:
%% $E ::=$
%% \SPACERF$v$
%% \SPACERP$x$
%% \SPACERP$\LAMBDA{x}{E}$
%% \SPACERP$\APPLY{E_1}{E_2}$
%% \SPACERP$\IFIN{E_1}{\VALUES}{E_2}{E_3}$
%% \SPACERP$\PROMPT{E}$
%% \SPACERP$\CONTROL{x}{E}$

\def\CITE{\TODOQ{[Cite something]}}
\def\union{\cup}
\newcommand{\DISJOINTUNION}[0]{\uplus}

\newcommand{\EMPTYLIST}[0]{\texttt{()}}
%% \newcommand{\CONS}[2]{{\tt (}$#1${\tt\ .\ }$#2${\tt )}}
%% \newcommand{\SINGLETON}[1]{{\tt (}$#1${\tt )}}
\newcommand{\CONS}[2]{{\texttt{(}}{#1}{\texttt{.}}{#2}{\texttt{)}}}
%\newcommand{\CONS}[2]{${\tt (}${#1}${\tt \ .\ }${#2}${\tt )}$}
\newcommand{\SINGLETON}[1]{{\texttt{(}}#1{\texttt{)}}}
\newcommand{\SLIST}[1]{\texttt{(}#1\texttt{)}}

\newcommand{\ALAMBDA}[2]
  {[\lambda{#1}.{#2}]}
\newcommand{\AAPPLY}[2]
  {[{#1}\ @\ {#2}]}

\newcommand{\BUNDLENAME}[0]
  {\CODE{bundle}\xspace}
\newcommand{\LETNAME}[0]
  {\CODE{let}\xspace}
\newcommand{\PRIMITIVENAME}[0]
  {\CODE{primitive}\xspace}
\newcommand{\CALLNAME}[0]
  {\CODE{call}\xspace}
\newcommand{\CALLINDIRECTNAME}[0]
  {\CODE{call-{\allowbreak}indirect}\xspace}
\newcommand{\IFNAME}[0]
  {\CODE{if}\xspace}
\newcommand{\THENNAME}[0]
  {\CODE{then}\xspace}
\newcommand{\ELSENAME}[0]
  {\CODE{else}\xspace}
\newcommand{\FORKNAME}[0]
  {\CODE{fork}\xspace}
\newcommand{\JOINNAME}[0]
  {\CODE{join}\xspace}
\newcommand{\EXTENDEDNAME}[0]
  {\CODE{join}\xspace}
%% \newcommand{\DEFINENONPROCEDURENAME}[0]
%%   {\CODE{non-procedure}\xspace}%{\CODE{define-value}}
\newcommand{\DEFINEPROCEDURENAME}[0]
  {\CODE{procedure}\xspace}%{\CODE{define-procedure}}

%% \newcommand{\ACONSTANT}[1]
%%   {{#1}}
%% \newcommand{\AVARIABLE}[1]
%%   {{#1}}
%% \newcommand{\ABUNDLE}[1]
%%   {[\CODE{bundle}\ {#1}]}
%% \newcommand{\APRIMITIVE}[2]
%%   {[\pi\ {#1}\ {#2}]}
%% \newcommand{\ACALL}[2]
%%   {[{#1}\ {#2}]}
%% \newcommand{\AIFIN}[4]
%% %  {[{#1} \in {#2} \rightarrow {#3},\ {#4}]}
%%   {[\CODE{if}\ {#1} \in \{{#2}\}\ \CODE{then}\ {#3}\ \CODE{else}\ {#4}]}

%% \newcommand{\ALET}[3]
%%   {[{\tt let}\ {#1}\ {\tt be}\ {#2}\ {\tt in}\ {#3}]}
%% \newcommand{\AFORK}[2]
%%   {[\FORK{#1\ #2}]}
%% \newcommand{\AJOIN}[1]
%%   {[\JOIN{#1}]}
%% \newcommand{\AEXTENDED}[2]
%%   {[\chi\ {#1}\ {#2}]}
%% \newcommand{\ADEFINENONPROCEDURE}[2]
%%   {[\CODE{define-value}\ {#1}\ {#2}]} % \triangleq
%% \newcommand{\ADEFINEPROCEDURE}[3]
%%   {[\CODE{define-procedure}\ {\ACALL{#1}{#2}}\ {#3}]}

%% %% \newcommand{\FORK}[0]
%% %%   {\downpitchfork}
%% %% \newcommand{\JOIN}[0]
%% %%   {\uppitchfork}

%% \newcommand{\AFUTURE}[1]
%%   {[\CODE{fork} {#1}]}
%% \newcommand{\ATOUCH}[1]
%%   {[\CODE{join} {#1}]}

%% \newcommand{\ABOX}[1]
%%   {[box\ {#1}]}
%% \newcommand{\AUNBOX}[1]
%%   {[unbox\ {#1}]}
%% \newcommand{\ASETX}[2]
%%   {[set!\ {#1} {#2}]}
%% \newcommand{\ASET}[2]
%%   {\ASETX{#1}\ {#2}}

%% %\newcommand{\ASEQUENCE}[2]
%% %  {{#1};\ {#2}}

%% \newcommand{\ADEFINE}[2]
%%   {{#1} \triangleq {#2}}
%% \newcommand{\ADEFINEMACRO}[3]
%%   {\CONS{#1}{#2}\ \equiv\ #3}
%% %  {${\tt (}$#1${\tt \ .\ }$#2${\tt )}$\ \equiv\ #3}
%% \newcommand{\VALUEPLUS}[0]
%%   {v^{+}}
%% %% \newcommand{\APPLY}[2]
%% %%   {{#1}@{#2}}
%% \newcommand{\COMPOSE}[2]
%%   {{#1}\circ{#2}}
%% %\newcommand{\SEQUENCE}[2]
%% %  {{#1};{#2}}
%% \newcommand{\ALAMBDATWO}[3]
%%   {\ALAMBDA{#1}{\ALAMBDA{#2}{#3}}}
%% \newcommand{\ALAMBDATHREE}[4]
%%   {\ALAMBDA{#1}{\ALAMBDA{#2}{\ALAMBDA{#3}{#4}}}}
%% \newcommand{\ALAMBDAFOUR}[5]
%%   {\ALAMBDA{#1}{\ALAMBDA{#2}{\ALAMBDA{#3}{\ALAMBDA{#4}{#5}}}}}
%% \newcommand{\ASHIFT}[2]
%%   {[\xi{#1}.{#2}]}
%% \newcommand{\ARESET}[1]
%%   {[\langle{#1}\rangle]}
%% \newcommand{\ACONTROL}[2]
%%   {[{\mathcal{C}}{#1}.{#2}]}
%% \newcommand{\APROMPT}[1]
%%   {[\#{#1}]}

\newcommand{\BLACK}[1]
  {\textcolor{black}{#1}}
\newcommand{\RED}[1]
  {\textcolor{red}{#1}}
\newcommand{\PURPLE}[1]
  {\textcolor{purple}{#1}}
\newcommand{\BROWN}[1]
  {\textcolor{brown}{#1}}
\newcommand{\YELLOW}[1]
  {\textcolor{yellow}{#1}}
\newcommand{\BLUE}[1]
  {\textcolor{blue}{#1}}
\newcommand{\GREEN}[1]
  {\textcolor{darkgreen}{#1}}
\newcommand{\DEF}[0]
  {\stackrel{\text{\tiny def}}{=}}
\newcommand{\BELONGS}[0]
  {\ \epsilon\ }
\newcommand{\BELONGSREVERSE}[0]
  {\ \backepsilon\ }

\newcommand{\SEMANTICE}[1]
  {\LFATBRACKETS{E}{#1}}
\newcommand{\SEMANTICESTAR}[1]
  {\LFATBRACKETS{E^{*}}{#1}}
\newcommand{\SEMANTICS}[1]
  {\LFATBRACKETS{S}{#1}}
\newcommand{\SEMANTICSSTAR}[1]
  {\LFATBRACKETS{S^{*}}{#1}}
\newcommand{\SEMANTICT}[1]
  {\LFATBRACKETS{T}{#1}}
\newcommand{\SEMANTICTSTAR}[1]
  {\LFATBRACKETS{T^{*}}{#1}}
\newcommand{\SEMANTICPRIMITIVE}[2]
  {\LFATBRACKETS{P}{{#1}, {#2}}}
\newcommand{\SEMANTICAPPLY}[2]
  {\LFATBRACKETS{A}{#1, #2}}
\newcommand{\SEMANTICM}[1]
  {\LFATBRACKETS{M}{#1}}
\newcommand{\SEMANTICMADDEND}[2]
  {\mbox{$\mathcal{M}_{#1}[\![ #2 ]\!]$}}
\newcommand{\SEMANTICMACROCALL}[3]
  {\mbox{$\mathcal{MC}[\![ #1, #2, #3 ]\!]$}}
\newcommand{\POSSIBLY}[0]{{\textcolor{red}{[{\bf The thing will {\em possibly} be like this}]}}}
\newcommand{\EPSILON}[0]{$\varepsilon$\xspace}
\newcommand{\EPSILONZERO}[0]{$\varepsilon_{0}$\xspace}
\newcommand{\EPSILONONE}[0]{$\varepsilon_{1}$\xspace}
\newcommand{\EPSILONNOXSPACE}[0]{$\varepsilon$}
\newcommand{\EPSILONZERONOXSPACE}[0]{$\varepsilon_{0}$}
%\newcommand{\EPSILONZERO}[0]{$\varepsilon_{0}$}
\newcommand{\EPSILONZEROHOLE}[0]{$\varepsilon_{0}^{\HOLE}$\xspace}
\newcommand{\EPSILONMINUSONE}[0]{$\varepsilon_{-1}$\xspace}
\newcommand{\LAMBDA}[0]{$\lambda$}
\newcommand{\LAMBDACALCULUS}[0]{\mbox{$\lambda$-calculus}\xspace}
\newcommand{\PICALCULUS}[0]{\mbox{$\pi$-calculus}\xspace}

\newcommand{\HOLEDES}[0]{\ensuremath{\SET{E}_{\HOLE}}}

\newcommand{\iem}[1]{\index{#1}{\em{#1}}}
\newcommand{\textiti}[1]{\index{#1}{\textit{#1}}}
\newcommand{\ind}[1]{\index{#1}{#1}}
\newcommand{\idef}[1]{\index{#1}{\sl{#1}}}
\newcommand{\TDEF}[1]{\idef{#1}}

\newcommand{\QUOTATION}[3]
      {\begin{quotation}
        {\em #1}\\
        {\INVISIBLE{}\hfill---~{#2}, {#3}}
      \end{quotation}}
\newcommand{\LATER}[1]{\textcolor{blue}{[{\bf Later:} {#1}]}}
\newcommand{\UPDATE}[1]{\textcolor{purple}{[{\bf Update:} {#1}]}}

\newcommand{\TO}[0]
  {\ \to\ }
\newcommand{\TOENEW}[0]
%  {\ \longrightarrow_{\SET{E}}\ }
  {\Longrightarrow_{\SET{E}}}
\newcommand{\TOE}[0]
%  {\ \longrightarrow_{\SET{E}}\ }
  {\LINEBREAK\longrightarrow_{\SET{E}}\LINEBREAK}
\newcommand{\TOEP}[0]
%  {\ \longrightarrow_{\SET{E}}\ }
  {\LINEBREAK\longrightarrow^{+}_{\SET{E}}\LINEBREAK}
%% \newcommand{\TOET}[0]
%%   {\ \longrightarrow_{\SET{E}}^{+}\ }
%% \newcommand{\TOEST}[0]
%%   {\ \longrightarrow_{\SET{E}}^{\slashed{\parallel}+}\ }
%% \newcommand{\TOESRT}[0]
%%   {\ \longrightarrow_{\SET{E}}^{\slashed{\parallel}*}\ }
%% \newcommand{\TOERT}[0]
%%   {\ \longrightarrow_{\SET{E}}^{*}\ }
%% \newcommand{\TOES}[0]
%%   {\longrightarrow_{\SET{E}}^{\slashed{\parallel}}}
\newcommand{\TOET}[0]
  {\LINEBREAK\longrightarrow_{\SET{E}}^{+}\LINEBREAK}
\newcommand{\TOEST}[0]
  {\LINEBREAK\longrightarrow_{\SET{E}}^{\slashed{\parallel}+}\LINEBREAK}
\newcommand{\TOESRT}[0]
  {\LINEBREAK\longrightarrow_{\SET{E}}^{\slashed{\parallel}*}\LINEBREAK}
\newcommand{\TOERT}[0]
  {\LINEBREAK\longrightarrow_{\SET{E}}^{*}\LINEBREAK}
\newcommand{\TOES}[0]
  {\longrightarrow_{\SET{E}}^{\slashed{\parallel}}}

\newcommand{\REDUCES}[0]
  {\twoheadrightarrow_{\SET{E}}}
\newcommand{\DOESNOTREDUCE}[0]
  {\ensuremath{\mathrel{\slashed{\REDUCES}}}}
\newcommand{\TOEFT}[0]
  {\ \longrightarrow_{e,\FAILURE}^{*}\ }
\newcommand{\EVENTUALLYFAILSBECAUSEOF}[1]
  {\Downarrow_{\SET{E}}\FAILURE_{#1}}
%\newcommand{\DOESNTFAILBECAUSEOF}[1]
%  {{\slashed{\Downarrow}}_{\FAILUREONARROW}^{#1}}
\newcommand{\EVENTUALLYFAILSBECAUSEOFDIMENSION}[0]
  {\EVENTUALLYFAILSBECAUSEOF{\#}}
\newcommand{\DOESNTFAILBECAUSEOFDIMENSION}[0]
  {\DOESNTFAILBECAUSEOF{\#}}  
\newcommand{\EVENTUALLYFAILSBECAUSEOFPRIMITIVE}[0]
  {\EVENTUALLYFAILSBECAUSEOF{\SET{P}}}
\newcommand{\DOESNTFAILBECAUSEOFPRIMITIVE}[0]
  {\DOESNTFAILBECAUSEOF{\SET{P}}}  
\newcommand{\EVENTUALLYFAILSBECAUSEOFENVIRONMENTS}[0]
  {\EVENTUALLYFAILSBECAUSEOF{\SET{X}}}

\newcommand{\EVENTUALLYFAILSBECAUSEOFWITHRELATION}[2]
  {\Downarrow_{\SET{E}}^{#2}\FAILURE_{#1}}

\newcommand{\EVENTUALLYFAILSSBECAUSEOF}[1]
  {\Downarrow_{\SET{E}}^{\slashed{\parallel}}\FAILURE_{#1}}
\newcommand{\EVENTUALLYFAILSSBECAUSEOFDIMENSION}[0]
  {\EVENTUALLYFAILSSBECAUSEOF{\#}}
\newcommand{\EVENTUALLYFAILSSBECAUSEOFPRIMITIVE}[0]
  {\EVENTUALLYFAILSSBECAUSEOF{\SET{P}}}
\newcommand{\EVENTUALLYFAILSSBECAUSEOFENVIRONMENTS}[0]
  {\EVENTUALLYFAILSSBECAUSEOF{\SET{X}}}
%  {\EVENTUALLYFAILSBECAUSEOF{\textcolor{red}{\rho}}}
\newcommand{\EVENTUALLYFAILSS}[0]
  {\EVENTUALLYFAILSSBECAUSEOF{}}
\newcommand{\DOESNTFAILBECAUSEOFENVIRONMENTS}[0]
%  {\DOESNTFAILBECAUSEOF{\textcolor{red}{\rho}}}
  {\DOESNTFAILBECAUSEOF{\SET{X}}}
%\newcommand{\EVENTUALLYFAILSBECAUSEOFANOTHERTHREAD}[0]
%  {\EVENTUALLYFAILSBECAUSEOF{\parallel}}
%\newcommand{\DOESNTFAILBECAUSEOFANOTHERTHREAD}[0]
%  {\DOESNTFAILBECAUSEOF{\parallel}}
\newcommand{\EVENTUALLYFAILS}[0]
  {\EVENTUALLYFAILSBECAUSEOF{}}
\newcommand{\DOESNTFAIL}[0]
  {{\slashed{\Downarrow}_{\FAILUREONARROW}}}

\newcommand{\FAILUREBECAUSEOFDIMENSION}[0]
  {\FAILURE^{\#}}
\newcommand{\FAILUREBECAUSEOFPRIMITIVE}[0]
  {\FAILURE^{\Pi}}
\newcommand{\FAILUREBECAUSEOFENVIRONMENTS}[0]
  {\FAILURE^{\textcolor{red}{\rho}}}

\newcommand{\FAILURE}[0]
%  {\textcolor{red}{\divideontimes}}
%  {\textbf{E}}
%  {\maltese} 
%  {\text{$\skull$}}
%  {\text{\small$\skull$}} % It looks the same with and without \small
%  {\text{\tiny$\skull$}} 
%  {\frownie}
%  {\rip}
%  {\textrm{\tiny{\blitza}}}
%  {\frownie}
%  {\textrm{\lightning}}
%  {\textrm{\rm \textinterrobang}}
%  {\textit{\textinterrobang}}
%  {\textrm{\blitza}}
%  {\textrm{\rm \textreferencemark}}
  {\textrm{\rm \large \textreferencemark}}
\newcommand{\FAILUREONARROW}[0]
  {\FAILURE}
%  {\textcolor{red}{\divideontimes}}
%  {\textbf{E}}
%  {\maltese} 
% {\skull}
% {{}^{\text{\tiny$\skull$}}}
% {{}^{\textrm{\blitza}}}
%{^{^{\text{\blitza}}}}

\newcommand{\FAILSBECAUSEOF}[1]
%  {\longrightarrow_{\SET{E}}\FAILURE^{#1}}
  {\longrightarrow_{\SET{E}}\FAILURE_{#1}}
\newcommand{\DOESNTFAILBECAUSEOF}[1]
%  {\longrightarrow_{\SET{E}}\FAILURE^{#1}}
  {{\slashed{\longrightarrow}}_{\SET{E}}\FAILURE_{#1}}
\newcommand{\FAILSBECAUSEOFDIMENSION}[0]
  {\FAILSBECAUSEOF{\#}}
\newcommand{\FAILSBECAUSEOFPRIMITIVE}[0]
  {\FAILSBECAUSEOF{\SET{P}}}
\newcommand{\FAILSBECAUSEOFENVIRONMENTS}[0]
  {\FAILSBECAUSEOF{\SET{X}}}
%\newcommand{\FAILSBECAUSEOFANOTHERTHREAD}[0]
%  {\FAILSBECAUSEOF{\parallel}}
\newcommand{\FAILS}[0]
  {\FAILSBECAUSEOF{}}

\newcommand{\DIVERGESE}[0]
  {\Uparrow_{\SET{E}}}
%\newcommand{\DIVERGESES}[0]
%  {\Uparrow_{\SET{E}}^{\slashed{\parallel}}}
\newcommand{\DOESNTDIVERGE}[0]
  {{\slashed{\uparrow}}}
\newcommand{\CONVERGES}[0]
  {\Downarrow}
\newcommand{\DOESNTCONVERGE}[0]
  {{\slashed{\CONVERGES}}}
\newcommand{\CONVERGESP}[0]
  {\CONVERGES_{\SET{P}}}
\newcommand{\CONVERGEST}[0]
  {\CONVERGES_{\SET{T}}}
\newcommand{\CONVERGESPPROVISIONAL}[0]
  {\CONVERGES_{\SET{P}}^{\thicksim}}
\newcommand{\CONVERGESE}[0]
  {\CONVERGES_{\SET{E}}}
\newcommand{\CONVERGESI}[0]
  {\CONVERGES_{\SET{I}}}
\newcommand{\TOITILDE}[0]
  {\longrightarrow_{\SET{I}}^{_\thicksim}}
\newcommand{\TOPTILDE}[0]
  {\longrightarrow_{\SET{P}}^{_\thicksim}}
\newcommand{\CONVERGESITILDE}[0]
  {\CONVERGES^{\thicksim}_{\SET{I}}}
\newcommand{\CONVERGESES}[0]
  {\CONVERGES_{\SET{E}}^{\slashed{\parallel}}}
\newcommand{\CONVERGESSE}[0]
  {\CONVERGESES}
\newcommand{\DOESNTCONVERGEE}[0]
  {{\slashed{\CONVERGES}}_{\SET{E}}}
\newcommand{\NCONVERGESE}[0]
  {\DOESNTCONVERGEE}
\newcommand{\TOEE}[0]
  {\ \longrightarrow_{ee}\ }
%  {\ \to_{EE}\ }

\newcommand{\TOD}[0]
  {\longrightarrow_{\SET{D}}}
\newcommand{\TOT}[0]
  {\longrightarrow_{\SET{T}}}
\newcommand{\TOP}[0]
  {\longrightarrow_{\SET{P}}}
\newcommand{\TOV}[0]
  {\longrightarrow_{\SET{V}}}
\newcommand{\TOTPROVISIONAL}[0]
  {\xrightarrow{provisional}_{\SET{T}}}


\newcommand{\TOM}[0]
  {\to_M}
\newcommand{\TOBIG}[0]
  {\downarrow}
\newcommand{\TOBIGE}[0]
  {\TOBIG_E}
\newcommand{\TOBIGT}[0]
  {\TOBIG_T}
\newcommand{\TOBIGP}[0]
  {\TOBIG_P}
\newcommand{\TOBIGM}[0]
  {\TOBIG_M}
\newcommand{\PARALLELCONFIGURATION}[2]
  {{#1} \rhd {#2}}
\newcommand{\PCONF}[2]
  {\PARALLELCONFIGURATION{#1}{#2}}
%\newcommand{\REWRITE}[3]
%  {{#1}\ \vdash\ {#2}\ \TOE\ {#3}}
\newcommand{\REWRITEP}[5]
%  {{\PCONF{#1}{#2}}\ \TOE\ {\PCONF{#3}{#4}}}
  {{#1}\ \vdash\ {\PCONF{#2}{#3}}\ \TOE\ {\PCONF{#4}{#5}}}
\newcommand{\REWRITEBIGE}[5]
  {{#1}\ \vdash\ \PCONF{#2}{#3}\ \TOBIGE\ \PCONF{#4}{#5}}
\newcommand{\REWRITEM}[4]
  {\PCONF{#1}{#2}\ \TOM\ \PCONF{#3}{#4}}
%% \newcommand{\STATE}[1]
%%   {\STATEQ_{#1}}
%% \newcommand{\STATEDEFAULT}[0]
%%   {\STATE{\Gamma, \Pi, M}}
%% \newcommand{\STATEQ}[0]
%%   {\Sigma}

\newcommand{\FUTURES}[0]
  {\Phi}

\newcommand{\RULETWO}[5]
  {\begin{prooftree}
      \LeftLabel{$[#1]$}
      \RightLabel{$#5$}
      \AxiomC{$#2$}
      \AxiomC{$#3$}
      \BinaryInfC{$#4$}
  \end{prooftree}}

\newcommand{\RULETHREE}[6]
  {\begin{prooftree}
      \LeftLabel{$[#1]$}
      \RightLabel{$#6$}
      \AxiomC{$#2$}
      \AxiomC{$#3$}
      \AxiomC{$#4$}
      \TrinaryInfC{$#5$}
  \end{prooftree}}

\newcommand{\RULEONE}[4]
  {\begin{prooftree}
      \LeftLabel{$[#1]$}
      \RightLabel{$#4$}
      \AxiomC{$#2$}
      \UnaryInfC{$#3$}
  \end{prooftree}}
  
\newcommand{\RULEZERO}[3]
  {\RULEONE{#1}{\INVISIBLE{hack}}{#2}{#3}}

\newcommand{\SET}[1]
  {\mathbb{#1}}
\newcommand{\VALUES}[0]
  {\SET{V}}
\newcommand{\NATURALS}{\SET{N}}
\newcommand{\INTEGERS}{\SET{Z}}
\newcommand{\RATIONALS}{\SET{Q}}
\newcommand{\REALS}{\SET{R}}
\newcommand{\SYMBOLS}{\SET{S}}

\newcommand{\UNION}{\cup}
\newcommand{\INTERSECTION}{\cap}

\newcommand{\CLOSURE}[3]
  {\mathcal{C}({#1}, {#2}, {#3})}

\newcommand{\NONVALUE}[1]
  {{#1} \notin \VALUES}
%\newcommand{\VALUE}[1] {{#1} \in \VALUES}

\newcommand{\BULLET}[0]
  {\bullet}
\newcommand{\SECTION}[0]
  {§}
\newcommand{\SECTIONS}[0]
  {§§}
\newcommand{\SOMEREF}[0]
  {\textcolor{red}{\textbf{[\SECTION???]}}}
\newcommand{\REFSOMETHING}[0]
  {\SOMEREF}

%\newcommand{\UNFILLEDUNPOSITIONEDQED}{\ensuremath{\Box}}
\newcommand{\UNFILLEDUNPOSITIONEDQED}{\ensuremath{\square}}
\newcommand{\FILLEDUNPOSITIONEDQED}{\ensuremath{\blacksquare}}
%\newcommand{\FILLEDUNPOSITIONEDQED}{\ensuremath{\filledsquare}}

%\newcommand{\qed}{\hfill \ensuremath{\Box}}
\newcommand{\UNFILLEDQED}{\hfill \ensuremath{\UNFILLEDUNPOSITIONEDQED}}
\newcommand{\FILLEDQED}{\hfill \ensuremath{\FILLEDUNPOSITIONEDQED}}

\newcommand{\EXAMPLEQED}{\UNFILLEDQED}
\newcommand{\PROOFQED}{\FILLEDQED}
\newcommand{\NOPROOF}{\PROOFQED}
\newcommand{\NOPROOFQED}{\NOPROOF}

\newcommand{\INJECT}{inject}
\newcommand{\EJECT}{eject} % Christophe didn't like this name. To do: find the name he
                           % liked; I have it on paper

%\newcommand{\CODE}[1]{\verb!{#1}!}
\newcommand{\CODE}[1]{{\rm \texttt{#1}}}
\newcommand{\FILE}[1]{\texttt{#1}}
%\newcommand{\UNIT}[0]{$\square$}
\newcommand{\UNIT}[0]{$\bullet$}
%\newcommand{\UNIT}[0]{$\star$}
\newcommand{\SYMBOL}[1]{\CODE{\#s-}{#1}}
\newcommand{\SYMBOLTEXT}[1]{\CODE{\#s-#1}}
\newcommand{\SYMBOLMETA}[1]{\CODE{\#s-{\ensuremath{#1}}}}

%\newcommand{\HASDIMENSION}{:_d}
\newcommand{\HASDIMENSION}{:_{\#}}

\newcommand{\LINEBREAK}[0]
  {\allowbreak}
%  {\linebreak[1]{\allowbreak{}}}
%{}
%%%%
\newcommand{\DHANDLE}[2]
  {\LINEBREAK{#2}_{#1}}
\newcommand{\DCONSTANT}[2]
  {\LINEBREAK{#2}_{#1}}
\newcommand{\DVARIABLE}[2]
  {\LINEBREAK{#2}_{#1}}
\newcommand{\DBUNDLE}[2]
  {\LINEBREAK[\BUNDLENAME\xspace\ \LINEBREAK{#2}]_{#1}}
\newcommand{\DPRIMITIVE}[3]
  {\LINEBREAK[\PRIMITIVENAME\ \LINEBREAK{{\tt #2}}\ \LINEBREAK{#3}]_{#1}}
\newcommand{\DCALL}[3]
%  {\LINEBREAK[\CALLNAME\ \LINEBREAK{{\tt \CODE{#2}}}\ \xspace\LINEBREAK{#3}]_{#1}}
  {\LINEBREAK[\CALLNAME\ \LINEBREAK{{{#2}}}\ \xspace\LINEBREAK{#3}]_{#1}}
\newcommand{\DCALLINDIRECT}[3]
%  {\LINEBREAK[\CALLINDIRECTNAME\ \LINEBREAK{{\tt \CODE{#2}}}\ \xspace\LINEBREAK{#3}]_{#1}}
  {\LINEBREAK[\CALLINDIRECTNAME\ \LINEBREAK{{{#2}}}\ \xspace\LINEBREAK{#3}]_{#1}}
\newcommand{\DIFIN}[5]
%  {[{#2} \in {#3} \rightarrow {#4},\ {#5}]_{#1}}
  {\LINEBREAK[\IFNAME\ \LINEBREAK{#2} \LINEBREAK\in \LINEBREAK\{{#3}\}\ \LINEBREAK\CODE{then}\ \LINEBREAK{#4}\ \LINEBREAK\CODE{else}\ \LINEBREAK{#5}]_{#1}}
\newcommand{\DLET}[4]
  {\LINEBREAK[\LETNAME\ \LINEBREAK{#2}\ \LINEBREAK{\tt be}\ \LINEBREAK{#3}\ \LINEBREAK{\tt in}\ \LINEBREAK{#4}]_{#1}}
\newcommand{\DFORK}[3]
  {\LINEBREAK[\FORKNAME\ \LINEBREAK{#2}\ \LINEBREAK{#3}]_{#1}}
\newcommand{\DJOIN}[2]
  {\LINEBREAK[\JOINNAME\ \LINEBREAK{#2}]_{#1}}
\newcommand{\DEXTENDED}[3]
  {\LINEBREAK[\EXTENDEDNAME\ \LINEBREAK{#2}\ \LINEBREAK{#3}]_{#1}}
\newcommand{\DDEFINENONPROCEDURE}[3]
%  {\LINEBREAK[\DEFINENONPROCEDURENAME\ \LINEBREAK{{\tt #2}}\ \LINEBREAK{#3}]_{#1}}
  {\LINEBREAK[\DEFINENONPROCEDURENAME\ \LINEBREAK{#2}\ \LINEBREAK{#3}]_{#1}}
\newcommand{\DDEFINEPROCEDURE}[3]
%  {\LINEBREAK[\DEFINEPROCEDURENAME\CODE{\ \LINEBREAK(}{{\tt \CODE{#2}}\ \xspace\LINEBREAK{#3}}\CODE{)\ \LINEBREAK}{#4}]_{#1}}
  {\LINEBREAK[\DEFINEPROCEDURENAME\CODE{\ \LINEBREAK(}{{{#1}}\ \xspace\LINEBREAK{#2}}\CODE{)\ \LINEBREAK}{#3}]}
\newcommand{\DCALLWITHHOLE}[2]
  {\LINEBREAK[\CODE{call}\ {#2}\ \LINEBREAK{\HOLE}]_{#1}}
\newcommand{\DFORKWITHHOLE}[2]
  {\LINEBREAK[\FORKNAME\ \LINEBREAK{#2}\ \LINEBREAK\HOLE]_{#1}}
\newcommand{\DBUNDLEWITHHOLE}[1]
  {\LINEBREAK[\BUNDLENAME\ \LINEBREAK\HOLE]_{#1}}
\newcommand{\DPRIMITIVEWITHHOLE}[2]
  {\LINEBREAK[\PRIMITIVENAME\ \LINEBREAK{#2}\ \LINEBREAK\HOLE]_{#1}}

\newcommand{\AHOLE}[0]
  {\bullet}

\newcommand{\SEQUENCE}[1]
  {\langle {#1} \rangle}
%  {\text{$\langle${}${#1}${}$\rangle$}}
\newcommand{\EMPTYSEQUENCE}[0]
  {\SEQUENCE{}}

\def\DRSH{\ensuremath{\mathrel{\slashed{\drsh}}}}
\def\NDRSH{\ensuremath{\mathrel{\slashed{\drsh}}}}
\newcommand{\ISTAIL}[1]
  {{#1}\DRSH}
%  {{#1}\Rsh}
\newcommand{\ISTAILFOR}[2]
  {\ISTAIL{#1}{#2}}
\newcommand{\ISNOTTAIL}[1]
  {{#1}\NDRSH}

\newcommand{\DIMENSION}[1]
  {\mbox{$d \ldbrack {#1}\rdbrack$}}

\newcommand{\MEET}[0]
  {\wedge}
\newcommand{\JOIN}[0]
  {\vee}
\newcommand{\SQMEET}[0]
  {\sqcap}
\newcommand{\SQJOIN}[0]
  {\sqcup}
\newcommand{\BIGSQJOIN}[0]
  {\bigsqcup}
\newcommand{\BIGSQMEET}[0]
  {\bigsqcap}
%% \newcommand{\BIGMEET}[2]
%%   {\bigwedge{#2}}
%% \newcommand{\BIGJOIN}[2]
%%   {\bigvee{#2}}

\newcommand{\SUCHTHAT}[0]
%  {~|~}
%  {~.~}
  {:}

\newcommand{\SQL}[0]
  {\sqsubset}
\newcommand{\SQLE}[0]
  {\sqsubseteq}
\newcommand{\SQGE}[0]
  {\sqsupseteq}
\newcommand{\NSQL}[0]
  {\nsqsubset}
\newcommand{\NSQLE}[0]
  {\nsqsubseteq}

\newcommand{\NATURALSB}[0]
  {\NATURALS_{\bot}}
\newcommand{\NATURALSTB}[0]
  {\NATURALS_{\bot}^{\top}}
\newcommand{\NATURALSBT}[0]
  {\NATURALSTB}

\newcommand{\EXPRESSIONS}{\SET{E}}
\newcommand{\PROGRAMS}{\SET{P}}
\newcommand{\CONSTANTS}{\SET{C}}
\newcommand{\STATESSET}{\SET{\Gamma}}
%\newcommand{\PROCEDURES}{\ensuremath{Proc}}
\newcommand{\PROCEDURES}{\SET{F}}

\newcommand{\PARTIAL}{\rightharpoonup}

\newcommand{\HOLE}{\Box}

% A way of superimposing two symbols, in Math mode.  I took the general idea
% from ``The LaTeX comprehensive symbol list'' and generalized:
\newcommand*{\SUPERIMPOSE}[2]
            {\mathrel{\mathchoice{\hbox{\hbox to 0pt{$\displaystyle{#1}$\hss}$\displaystyle{#2}$}}
                {\hbox{\hbox to 0pt{$\textstyle{#1}$\hss}$\textstyle{#2}$}}
                {\hbox{\hbox to 0pt{$\scriptstyle{#1}$\hss}$\scriptstyle{#2}$}}
                {\hbox{\hbox to 0pt{$\scriptscriptstyle{#1}$\hss}$\scriptscriptstyle{#2}$}}}}

\newcommand{\OCCURS}{\SUPERIMPOSE{\Subset}{-}}

\newcommand{\COLOREDMINIPAGE}[2]
  {\textcolor{#1}{
     \begin{minipage}{\linewidth}
       #2
     \end{minipage}}}

\newcommand{\PARAMETER}[0]
  {\mathunderscore}

\newcommand{\xqed}[1]{%
  \leavevmode\unskip\penalty9999 \hbox{}\nobreak\hfill
  \quad\hbox{\ensuremath{#1}}}
\def\QEDDEFINITION{\xqed{\UNFILLEDUNPOSITIONEDQED}}
%\def\QEDNOPROOF{\xqed{\blacksquare}} % To do: it should be smaller
\def\QEDNOPROOF{\xqed{\UNFILLEDUNPOSITIONEDQED}}
\def\QEDPROOF{\xqed{\FILLEDUNPOSITIONEDQED}}
\def\QEDAXIOM{\xqed{\UNFILLEDUNPOSITIONEDQED}}
\def\QEDEXAMPLE{\xqed{\Diamond}}
\def\QEDIMPLEMENTATIONNOTE{\QEDDEFINITION}
\def\QEDSYNTACTICCONVENTION{\QEDIMPLEMENTATIONNOTE}

\newcommand{\EMPTYSET}[0]{\varnothing}
%\newcommand{\EMPTYSET}[0]{\emptyset}
%\newcommand{\EMPTYSTACK}[0]{\EMPTYSET}
\newcommand{\EMPTYSTACK}[0]{\EMPTYSEQUENCE}

%\newcommand{\VALUESEPARATOR}[0]{!}
\newcommand{\VALUESEPARATOR}[0]{\text{\ensuremath{\wr}}} % This hack suppresses the spaces around \wr
%\newcommand{\ACTIVATIONSEPARATOR}[0]{?}
%\newcommand{\VALUESEPARATOR}[0]{\ensuremath{\wr}}
%\newcommand{\VALUESEPARATOR}[0]{\ensuremath{\dagger}}
\newcommand{\ACTIVATIONSEPARATOR}[0]{\ensuremath{\ddagger}}

%% \newcommand{\VALUESEPARATOR}[0]{\ensuremath{\lrcorner}}
%% \newcommand{\ACTIVATIONSEPARATOR}[0]{\ensuremath{\urcorner}}

\newcommand{\UNIFY}[0]
  {\equiv}
\newcommand{\UNIFIES}[0]
  {\UNIFY}
\newcommand{\DOESNTUNIFY}[0]
  {\ensuremath{\mathrel{\slashed{\equiv}}}}
  %{\nequiv}
\newcommand{\DOESNOTUNIFY}
  {\DOESNTUNIFY}

\newcommand{\SIMULATES}
  {\thickapprox}
%  {\thicksim}

\newcommand{\REDUCTIONEQUIVALENT}[0]
  {\equiv_{\EXPRESSIONS}}
\newcommand{\FILLBYHAND}[0]
  {\INVISIBLE{a\\\\a}}

\newcommand{\STATE}
  {state\xspace}
%  {context\xspace}
\newcommand{\STATES}
  {states\xspace}
%  {contexts\xspace}
\newcommand{\SSTATE}
  {State\xspace}
%  {Context\xspace}
\newcommand{\SSTATES}
  {States\xspace}
%  {Contexts\xspace}
\newcommand{\STATEKEY}
  {state key\xspace}
%  {context\xspace}
\newcommand{\STATEKEYS}
  {state keys\xspace}
%  {contexts\xspace}
\newcommand{\SSTATEKEY}
  {State key\xspace}
%  {Context\xspace}
\newcommand{\SSTATEKEYS}
  {State keys\xspace}
%  {Contexts\xspace}

\newcommand{\UPDATEENVIRONMENT}[3]
  {{#1}[{#2} \mapsto {#3}]}
\newcommand{\UPDATESTATE}[3]
  {{#1}[{}_{#2}^{#3}]}
\newcommand{\UPDATESTATEIN}[4]
  {\UPDATESTATE{#1}{#2}{{#3}~\mapsto~{#4}}}

%% % I want my unnumbered chapters to appear in the contents:
%% \newcommand{\UNNUMBEREDCHAPTER}[1]{\chapter*{#1}\addcontentsline{toc}{chapter}{\protect\numberline{}#1}}

%% % I want my unnumbered sections to appear in the contents.  They should be
%% % listed as chapters in the ToC, since they are independent top-level units.
%% %\newcommand{\UNNUMBEREDSECTION}[1]{\section*{#1}\addcontentsline{toc}{chapter}{\protect\numberline{}#1}}
%% \newcommand{\UNNUMBEREDSECTION}[1]{\section*{#1}\addcontentsline{toc}{section}{\protect\numberline{}#1}}
%% \newcommand{\UNNUMBEREDSUBSECTION}[1]{\subsection*{#1}\addcontentsline{toc}{subsection}{\protect\numberline{}#1}}
%% \newcommand{\UNNUMBEREDSUBSUBSECTION}[1]{\subsubsection*{#1}\addcontentsline{toc}{subsubsection}{\protect\numberline{}#1}}

% This use of \addstarredchapter is needed not to interfere with minitocs.
% I want my unnumbered chapters to appear in the contents:
%\newcommand{\UNNUMBEREDCHAPTER}[1]{\chapter*{#1}\addstarredchapter{\protect\numberline{}#1}}
\newcommand{\UNNUMBEREDCHAPTER}[1]{\chapter*{#1}\addstarredchapter{#1}}

% This use of \addcontentsline is needed not to interfere with minitocs.
% I want my unnumbered sections to appear in the contents.  They should be
% listed as chapters in the ToC, since they are independent top-level units.
\newcommand{\UNNUMBEREDSECTION}[1]{\section*{#1}\addcontentsline{toc}{section}{\protect\numberline{}#1}}

% I like verbatim environments to always have small fonts:
%% \makeatletter 
%% \g@addto@macro\@verbatim\small
%% \makeatother 
%\fvset{gobble=2, fontsize=\tiny} % For fancyvrb
\newcommand{\EPSILONGC}[0]{\CODE{epsilongc}\xspace}

\newcommand{\FATBRACKETSWITH}[2]
%  {\mbox{${#1}[\![ {#2} ]\!]$}}
%  {E_{\SET{#1}}{[{#2}]}}
%  {E_{\SET{#1}}{({#2})}}
%  {E_{\SET{#1}}{\Lbag{#2}\Rbag}}
%  {E_{\SET{#1}}{\lbag{{#2}}\rbag}}
  {E_{\SET{#1}}\BANANABRACKETS{#2}}
%  {E_{\SET{#1}}{\lcorners{#2}\rcorners}}
%  {E_{\SET{#1}}{\llcorner{#2}\lrcorner}}
%  {E_{\SET{#1}}{\ulcorner{#2}\urcorner}}
%  {E_{\SET{#1}}{\llfloor{#2}\rrfloor}}
\newcommand{\LOTSA}[1]
%  {\overline{#1}}
%  {{#1}^{*}}
  {{#1}s}
\newcommand{\EXPANDE}[1]{\FATBRACKETSWITH{E}{#1}}
\newcommand{\EXPANDX}[1]{\FATBRACKETSWITH{X}{#1}}
\newcommand{\EXPANDCS}[1]{\FATBRACKETSWITH{\LOTSA{C}}{#1}}
\newcommand{\EXPANDXS}[1]{\FATBRACKETSWITH{\LOTSA{X}}{#1}}
\newcommand{\EXPANDES}[1]{\FATBRACKETSWITH{\LOTSA{E}}{#1}}

\newcommand{\BOOTSTRAPPHASE}[1]{{\em ({#1})}}
%\newcommand{\WHATEVER}[1]{\ensuremath{\overline{\CODE{#1}}}}
\newcommand{\WHATEVER}[1]{\CODE{#1}} % The definition above was a bad idea

\newcommand{\BETWEENTINYANDSMALL}{\fontsize{9}{11}\selectfont}
\newcommand{\LARGERTHANSMALL}{\fontsize{12}{14}\selectfont}

\usepackage{censor}

\usepackage{fancyhdr}                    % Fancy Header and Footer
\pagestyle{fancy}                       % Sets fancy header and footer
%\fancyfoot{\flushright \textit{Summer 2012}}
\fancyfoot{} % Delete current footer settings
%% \let\footruleORIG\footrule
%% \renewcommand{\footrule}{\color{black} \footruleORIG}
%% \renewcommand{\footrulewidth}{1.0pt}
%% \fancyfoot[LE]{\bf{Luca SAIU}, Summer 2012}
%% \fancyfoot[RO]{\bf \TITLE, \SUBTITLE\ --- PhD thesis}
%\fancyfoot{{\tiny [preliminary --- Luca Saiu]}}

%\renewcommand{\chaptermark}[1]{         % Lower Case Chapter marker style
%  \markboth{\chaptername\ \thechapter.\ #1}}{}} %

%\renewcommand{\sectionmark}[1]{         % Lower case Section marker style
%  \markright{\thesection.\ #1}}         %

\fancyhead[LE,RO]{\textcolor{black}{\bfseries\thepage}}    % Page number (boldface) in left on even pages and right on odd pages
\fancyhead[RE]{\textcolor{black}{\bfseries\nouppercase{\leftmark}}}      % Chapter in the right on even pages
\fancyhead[LO]{\textcolor{black}{\bfseries\nouppercase{\rightmark}}}     % Section in the left on odd pages

\let\headruleORIG\headrule
\renewcommand{\headrule}{\color{black} \headruleORIG}
\renewcommand{\headrulewidth}{1.0pt}
\usepackage{colortbl}
\arrayrulecolor{black}

\fancypagestyle{plain}{
  \fancyhead{}
  \fancyfoot{}
  \renewcommand{\headrulewidth}{0pt}
}

\fancypagestyle{unobstrusive}{
  \fancyhead{}
  \fancyfoot[LE,RO]{\textcolor{black}{\bfseries\thepage}}
\fancyfoot[RE]{\textcolor{black}{\bfseries\nouppercase{\leftmark}}}
\fancyfoot[LO]{\textcolor{black}{\bfseries\nouppercase{\rightmark}}}
  %% \fancyfoot[RE]{\textcolor{black}{\em\nouppercase{\leftmark}}}
  %% \fancyfoot[LO]{\textcolor{black}{\em\nouppercase{\rightmark}}}
%  \fancyfoot[C]{\textcolor{black}{\bfseries\thepage}}
  \renewcommand{\headrulewidth}{0pt}
  \renewcommand{\footrulewidth}{1.0pt}
}

\usepackage[left=1.5in,right=1.3in,top=1.1in,bottom=1.1in,includefoot,includehead,headheight=13.6pt]{geometry}
\renewcommand{\baselinestretch}{1.05}

% Table of contents for each chapter
\usepackage[nottoc, notlof, notlot]{tocbibind}
\usepackage{makeidx}

%% \setcounter{tocdepth}{3} % I want a very detailed toc, including subsubsections
%% \makeindex
%% \usepackage{minitoc}
%% \let\minitocORIG\minitoc
%% \renewcommand{\minitoc}{\minitocORIG \vspace{1.5em}}
%% \setcounter{minitocdepth}{1}
%% %\setcounter{minitocdepth}{5}
%% \mtcindent=15pt
%% % Use \minitoc where to put a table of contents
%% %\dominitoc

\usepackage{xcolor}
\pagecolor{white}

% My pdf code
%% \usepackage{ifpdf}
%% \ifpdf
%  \usepackage[pdftex]{graphicx}
%  \usepackage{graphicx}
%  \DeclareGraphicsExtensions{.jpg}
  \DeclareGraphicsExtensions{.ps,.eps,.jpg}
%  \usepackage[hidelinks=true,colorlinks=false,linkcolor=blue,filecolor=blue,citecolor=blue,urlcolor=red,a4paper,pagebackref,hyperindex=true]{hyperref}
%  \usepackage[hidelinks=false,colorlinks=true,linkcolor=blue,filecolor=blue,citecolor=blue,urlcolor=red,a4paper,pagebackref,hyperindex=true]{hyperref}
  \usepackage[colorlinks=true,linkcolor=darkblue,filecolor=darkblue,citecolor=darkblue,urlcolor=darkred,a4paper,pagebackref,hyperindex=true]{hyperref}
%% \else
%%   \usepackage{graphicx}
%%   \DeclareGraphicsExtensions{.ps,.eps}
%%   \usepackage[colorlinks=true,linkcolor=blue,filecolor=blue,citecolor=blue,urlcolor=red,a4paper,dvipdfm,pagebackref,hyperindex=true]{hyperref}
%% \fi
\graphicspath{{.}{images/}}

% Change this to change the informations included in the pdf file
% See hyperref documentation for information on those parameters
\hypersetup
{
%% bookmarksopen=true,
%pdftitle={\SHORTTITLE},
%pdftitle={\TTITLE},
pdftitle={\TITLE\ -\ \SUBTITLE},
pdfauthor={\AUTHOR}, 
pdfsubject={\TITLE\ -\ \SUBTITLE}, %subject of the document
%pdftoolbar=false, % toolbar hidden
%% pdftoolbar=true, % toolbar shown
%% pdfmenubar=true, %menubar shown
%% pdfhighlight=/O, %effect of clicking on a link
%% colorlinks=true, %couleurs sur les liens hypertextes
%% %colorlinks=false, %couleurs sur les liens hypertextes
pdfpagemode=None, %aucun mode de page
pdfpagelayout=SinglePage, %ouverture en simple page
pdffitwindow=true, %pages ouvertes entierement dans toute la fenetre
%% linkcolor=linkcol, %couleur des liens hypertextes internes
%% citecolor=citecol, %couleur des liens pour les citations
%% urlcolor=linkcol %couleur des liens pour les url
}

\usepackage[amsmath,standard,thmmarks,framed,thref]{ntheorem} % Shall I really use the amsmath option?
%% \usepackage{amsthm}
%% %\qedsymbol{$\pi$}
%\renewcommand{\qedsymbol}{\heartsuit}

% I want identifiers such as CHAPTERNO.index
%\renewtheorem{theorem}{Theorem}[chapter]
%\renewtheorem{definition}{Definition}[chapter]
% The following things share the same counter as theorems:
\renewtheorem{definition}[theorem]{Definition} 
\renewtheorem{lemma}[theorem]{Lemma}
\renewtheorem{corollary}[theorem]{Corollary}
\renewtheorem{proposition}[theorem]{Proposition}
\newtheorem{conjecture}[theorem]{Conjecture}
\newtheorem{axiom}[theorem]{Axiom}
\newtheorem{informaldefinition}[theorem]{Informal Definition}
\newtheorem{implementationnote}[theorem]{Implementation Note}
\newtheorem{syntacticconvention}[theorem]{Syntactic Convention}

\def\TheoremSymbol{\bullet}
\def\DefinitionSymbol{\bullet}
\def\PropositionSymbol{\bullet}

%\renewcommand{\theorem}{\arabic{chapter}.\arabic{theorem}}
%\newtheorem{definition}{Definition}[subsection]

%\newtheorem{theorem}{Theorem}[section]
%% \newtheorem{theorem}{Theorem}
%% \newtheorem{lemma}[theorem]{Lemma}
%% \newtheorem{proposition}[theorem]{Proposition}
%% \newtheorem{corollary}[theorem]{Corollary}
%% \newtheorem{definition}{Definition}
%% \newtheorem{proof}{Proof}
%\newenvironment{proof}[1][Proof]{\begin{trivlist}
%\item[\hskip \labelsep {\bfseries #1}]}{\PROOFQED \end{trivlist}}
%%\renewenvironment{definition}[1][Definition]{\begin{trivlist}
%%\item[\hskip \labelsep {\bfseries #1}]}{\end{trivlist}}
%% \newenvironment{example}[1][Example]{\begin{trivlist}
%% \item[\hskip \labelsep {\bfseries #1}]}{\EXAMPLEQED \end{trivlist}}
%% \newenvironment{remark}[1][Remark]{\begin{trivlist}
%% \item[\hskip \labelsep {\bfseries #1}]}{\end{trivlist}}


\usepackage{relsize} % Relative font size is handy
% Change some defaults for our fancy verbatim (capital ``Verbatim''
% environments):
\fvset{numbers=left, fontsize=\relsize{-1}, frame=lines}

\usepackage[bbgreekl]{mathbbol} % This changes the appearance of \SET stuff,
                                % but allows me to use many more symbols in
                                % \SET

\newcommand*\cleartoleftpage{%
  \clearpage
  \ifodd\value{page}\hbox{}\newpage\fi
}
\usepackage[multiple]{footmisc} % Place footnote markers before the punctuation following them
